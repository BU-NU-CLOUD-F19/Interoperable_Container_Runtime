\subsection*{CIS Chapter 5}
The ways in which a container is started governs a lot of security implications. It is possible
to provide potentially dangerous run-time parameters that might compromise the host and
other containers on the host. Verifying container run-time is thus very important. In this project we implement various
recommendations to assess the container run-time security that is provided through CIS Chapter 5.

We implement following benchmarks: 

 \subsubsection*{5.1 Do not disable AppArmor Profile (Scored)} 
AppArmor protects the Linux OS and applications from various threats by enforcing
security policy which is also known as AppArmor profile. You can create your own
AppArmor profile for containers or use the Docker's default AppArmor profile. This would
enforce security policies on the containers as defined in the profile.

\subsubsection*{5.2 Verify SELinux security options, if applicable (Scored)} SELinux provides a Mandatory Access Control (MAC) system that greatly augments the
default Discretionary Access Control (DAC) model. You can thus add an extra layer of safety
by enabling SELinux on your Linux host, if applicable.

\subsubsection*{5.3 Restrict Linux Kernel Capabilities within containers (Scored)} By default, Containers start with a restricted set of Linux Kernel Capabilities. It
means that any process may be granted the required capabilities instead of root access.
Using Linux Kernel Capabilities, the processes do not have to run as root for almost all the
specific areas where root privileges are usually needed. 
\begin{itemize}
    \item NET\_ADMIN
    \item SYS\_ADMIN
    \item SYS\_MODULE
\end{itemize}

\subsubsection*{5.6 Do not run ssh within containers (Scored)}Running SSH within the container increases the complexity of security management by
making it
difficult to manage access policies and security compliance for SSH server. Difficult to manage keys and passwords across various containers. Difficult to manage security upgrades for SSH server
It is possible to have shell access to a container without using SSH, the needlessly
increasing the complexity of security management should be avoided.


\subsubsection*{5.9 Do not share the host's network namespace (Scored)} This is potentially dangerous. It allows the container process to open low-numbered ports
like any other root process. It also allows the container to access network services like Dbus on the Docker host. Thus, a container process can potentially do unexpected things
such as shutting down the Docker host. You should not use this option.

\subsubsection*{5.10 Limit memory usage for container (Scored)}
By default, container can use all of the memory on the host. You can use memory limit
mechanism to prevent a denial of service arising from one container consuming all of the
host’s resources such that other containers on the same host cannot perform their intended
functions. Having no limit on memory can lead to issues where one container can easily
make the whole system unstable and as a result unusable.

\subsubsection*{5.11 Set container CPU priority appropriately (Scored)}
By default, CPU time is divided between containers equally. If it is desired, to control the
CPU time amongst the container instances, you can use CPU sharing feature. CPU sharing
allows to prioritize one container over the other and forbids the lower priority container to
claim CPU resources more often. This ensures that the high priority containers are served
better.

\subsubsection*{5.24 Confirm cgroup usage (Scored)} System administrators typically define cgroups under which containers are supposed to
run. 
At run-time, it is possible to attach to a different cgroup other than the one that was
expected to be used. This usage should be monitored and confirmed. By attaching to a
different cgroup than the one that is expected, excess permissions and resources might be
granted to the container and thus, can prove to be unsafe.

\subsubsection*{5.28 Use PIDs cgroup limit (Scored)} Attackers could launch a fork bomb with a single command inside the container. This fork
bomb can crash the entire system and requires a restart of the host to make the system
functional again. PIDs cgroup --pids-limit will prevent this kind of attacks by restricting
the number of forks that can happen inside a container at a given time.

