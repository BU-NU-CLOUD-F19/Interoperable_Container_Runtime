Virtualization of resources has emerged in recent decades as a means to run multiple OSes on the same hardware. This particularly serves a useful function as this allows multiple applications to coexist on the same server, enabling efficencies in computing such as server consolidation.

Traditional VMs virtualize hardware resources, which results in the VMs taking up more resources. As such, OS-level virtualization, or Containers, have been developed. By sharing OS resources, containers are lightweight and can be spun up quickly while taking up fewer resources. Docker, introduced in 2013, is a popular runtime to manage containers as it addresses end-to-end management. However, Docker was initially a monolith with features not inherently dependent on each other being bundled together. As a result, alternative runtimes such as CRI-O and contianerd exist which implement container management at varying levels. [1]

The Open Container Initiative (OCI, https://www.opencontainers.org) has been established to create a open standard for container use regardless of the runtime being used to manage the container. However, the OCI only specifies downloading image then unpacking that image into an OCI Runtime filesystem bundle.

It does not standardize lifecycle management of the containers, thus each container implements lifecycle functionality in a different manner. It also does not ensure that consistent standards for the security of containers are present.

In this project, we will study the differences in popular runtimes Docker, containerd, and crio. Our focus will be on developing a service focused on ensuring that Center for Internet Security (CIS) Benchmarks for Docker are satisfied across other runtimes to ensure consistent application of security principles irrespective of container runtime differeces. Long term, the goal is to implement a lifecycle mangement solution that enables a common management framework for controlling containers across enviornments. By implementing a service to validate standard secuirty checks across runtimes, we intend to provide a Proof of Concept that such a common lifecycle management is possible.

\subsection{Vision and Goals}
Currently if someone wishes to launch an image in a container or perform any other lifecycle management functions on it, they must be sure that the scripts are configured correctly for the target container. For instance, launching images in Docker differs from doing so in CRI-O or containerd. This locks individuals and businesses into whichever container runtime they started with unless they invest the time required to edit the configuration and their scripts which holds the commands for target container.

Our short term goal for this project is to enable the set of Container Runtime tests run in the CIS Docker 1.13.0 Benchmark across any container runtime. These tests are specified in Chapter 5 of the document; example checks include restricting Linux Kernel capabilities within containers, limiting memory usage, and avoiding directly exposing the host devices to the containers. Publishing a minumum viable framework for this purpose will enable users to run their security checks using a single script across the most popular containers.

The ultimate goal is to develop an interoperable container runtime tool that allows user to perform common container lifecycle management functions among different runtimes using a single framework (e.g. start/stop execution, ps).

\subsection*{Users/Personas of the Project}
The intended user is a software developer who is developing, testing, and managing applications across containers running on different runtimes.

Example Use Case: A software developer would like to launch an image in CRI-O instead of Docker, because he realizes that CRI-O is more adaptable with Kubernetes, and using this capability will provide this application a lot more scalibility. Presently, he needs to deal with changing all the continous-integration scripts in order to be able to test and deploy his application on this new container runtime. With our interoperable framework in place, the developer is at least able to run security checks on the new container runtime without changing their scripts beyond specifying the new target container. In this way, the user's workflow is simplified and can apply a standard across runtimes with minimal effort.

\subsection*{Scope}
The runtimes in scope for capatibility for this project will be Docker, and CRI-O. containerd is considered a runtime in scope as a stretch goal.

This project aims to esnure that the framework implements commands that satisfy the CIS Docker 1.13.0 Benchmark related to Container Runtimes across our in-scope runtimes. In doing so, users will be enabled to run their security checks with a single script rather than requiring separate suites for each runtime. The MVP will be considered to be implementing select benchmarks in consultation with our mentor (highlighted in bold in the below list). Implementation of the full suite is a stretch goal.

These benchmarks are specified in pp 126-180 of the Benchmark documentation